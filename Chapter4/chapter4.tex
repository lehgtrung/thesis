\chapter{Các Kết Quả Thực Nghiệm}
\ifpdf
    \graphicspath{{Chapter4/Chapter4Figs/PNG/}{Chapter4/Chapter4Figs/PDF/}{Chapter4/Chapter4Figs/}}
\else
    \graphicspath{{Chapter4/Chapter4Figs/EPS/}{Chapter4/Chapter4Figs/}}
\fi
\label{chap_4}
\begin{quote}
\textit{Trong chương này, chúng tôi trình bày các kết quả thí nghiệm để đánh giá các mô hình được tìm hiểu mà đã trình bày ở chương trước. Bộ dữ liệu được dùng để tiến hành các thí nghiệm là bộ WMT'14 English-German (bộ dữ liệu tiếng Anh-tiếng Đức của cuộc thi Dịch máy WMT năm 2014). Các kết quả thí nghiệm cho thấy khi huấn luyện mô hình mà không sử dụng cơ chế Attention thì kết quả đạt được rất thấp. Các kết quả cũng cho thấy rằng các mô hình Attention Toàn cục, Attention Cục bộ, phương pháp Input feeding cho kết quả được cải thiện một cách rõ rệt.}
\end{quote}
\section{Các thiết lập thực nghiệm}
Chúng tôi tiến hành các thực nghiệm trên bộ dữ liệu WMT' 14 English-German được cung cấp trên trang chủ của Nhóm Xử lý Ngôn ngữ Tự nhiên Đại học Stanford. Bộ dữ liệu này gồm các cặp câu được viết dưới dạng ngôn ngữ tự nhiên ở 2 ngôn ngữ là tiếng Anh và tiếng Đức. Tất cả mô hình sẽ được huấn luyện trên tập dữ liệu này. Tập dữ liệu có khoảng 4,5 triệu cặp câu (trong đó có khoảng 116 triệu từ tiếng Anh và khoảng 110 triệu từ tiếng Đức).

Dữ liệu được tiến hành tiền xử lý bằng cách thực hiện tách từ đối với mỗi câu. Bộ từ vựng cho mỗi ngôn ngữ được sử dụng cho các mô hình là bộ từ vựng có 50.000 từ xuất hiện nhiều nhất (có tần số lớn nhất) trong dữ liệu huấn luyện của mỗi ngôn ngữ đó. Những từ nào không nằm trong bộ từ vựng sẽ được gán cho kí hiệu $<unk>$.

Trong quá trình huấn luyện, đối với những câu có độ dài lớn hơn 50 từ, chúng tôi chỉ lấy 50 từ đầu tiên và bỏ những từ còn lại. Chúng tôi thực hiện sắp xếp tất cả câu theo chiều dài của câu giảm dần (những câu nào có chiều dài lớn nhất thì đứng đầu), sau đó lấy ngẫu nhiên các mini-batches từ những câu đã được sắp xếp. Với việc sắp xếp như vậy, tốc độ huấn luyện của mô hình được cải thiện và mô hình học được tốt hơn.

Chúng tôi sử dụng các mô hình LSTM với mỗi LSTM có 4 tầng. Mỗi tầng LSTM có kích thước trạng thái ẩn là 1000 (sử dụng Bi-LSTM nên mỗi chiều sẽ có kích thước trạng thái ẩn là 500) và số chiều của word embedding là 1000. Các tham số của mô hình được khởi tạo ngẫu nhiên với phân phối đều trong đoạn $[-0,1; 0,1]$. Thuật toán để cực tiểu hóa hàm chi phí là Stochastic Gradient Descent (SGD) với kích thước của mini-batch là 128 mẫu huấn luyện. Cách lập lịch cho hệ số học: huấn luyện 12 epochs; hệ số học ban đầu là 1,0; sau 8 epochs, hệ số học sẽ giảm đi 1 nửa sau mỗi epoch tiếp theo. Gradient của các tham số sẽ được chuẩn hóa nếu norm của chúng vượt quá 5,0. Mô hình còn sử dụng cơ chế dropout với xác suất tắt các nơ-ron $p = 0.2$. Mỗi câu ở ngôn ngữ nguồn khi được đưa vào mô hình thì sẽ được đảo ngược trật tự. Đối với các mô hình Attention Cục bộ, kích thước của sổ $D = 10$.

Chúng tôi sử dụng ngôn ngữ lập trình Python và framework PyTorch dành cho Học sâu \cite{pytorchworkshop}. PyTorch hỗ trợ việc cài đặt các thuật toán một cách thân thiện, tự nhiên giống như Python và còn hỗ trợ xử lí tính toán song song trên GPU (Graphical Processing Units) rất mạnh mẽ. GPU mà chúng tôi sử dụng để thực hiện các thực nghiệm là NVIDIA Titan V. Để có thể huấn luyện một mô hình, cần đến 3-5 ngày.

Để đánh giá chất lượng dịch của các mô hình đã được huấn luyện, chúng tôi sử dụng tập dữ liệu kiểm thử \textit{newstest\_2014.en} và \textit{newstest\_2014.de} của cuộc thi WMT'14 và độ đo được sử dụng để đành giá là BLEU (BiLingual Evaluation Understudy) cùng với Perplexity. Dữ liệu validation được sử dụng là tập dữ liệu kiểm thử \textit{newstest\_2013.en} và \textit{newstest\_2013.de} của cuộc thi WMT'13.

Mô hình cơ bản (Baseline) mà chúng tôi sử dụng để so sánh với các mô hình Attention là mô hình với kiến trúc Bộ mã hóa-Bộ giải mã mà không có sử dụng cơ chế Attention.

\section{Kết quả thực nghiệm}
Các kết quả của các mô hình được ghi trong bảng \ref{wmt14-results}:

% Please add the following required packages to your document preamble:
% \usepackage{multirow}
% Please add the following required packages to your document preamble:
% \usepackage{multirow}
\begin{table}[]
	\centering
	\begin{tabular}{|l|l|l|}
		\hline
		\multicolumn{1}{|c|}{\multirow{2}{*}{\textbf{Model}}} & \multicolumn{2}{l|}{\textbf{BLEU}} \\ \cline{2-3} 
		\multicolumn{1}{|c|}{}                                & \textbf{Ours}   & \textbf{Paper}   \\ \hline
		Baseline                                              & 15.04           & 14.0             \\ \hline
		Baseline + global (general)                           & 20.25           & 17.3             \\ \hline
		Baseline + global (dot)                               & 19.02           & 18.6             \\ \hline
		Baseline + global (dot) + input feed                  & 20.23           &                  \\ \hline
		Baseline + global (dot) + input feed + unk rpl        & 22.71           &                  \\ \hline
		Baseline + local-p (genenral) + input feed            & 20.75           &                  \\ \hline
	\end{tabular}
	\caption{Kết quả của các mô hình trên tập dữ liệu WMT'14 English-German.}
	\label{wmt14-results}
\end{table}
Từ bảng kết quả cho thấy việc sử dụng cơ chế Attention giúp cải thiện kết quả rất lớn. Mô hình Baseline có kết quả trên độ đo BLEU của chúng tôi là 15,04. Khi sử dụng cơ chế Attention, mô hình cho khoảng chênh lệch nhỏ nhất giữa mô hình Baseline và các mô hình Attention là mô hình Attention Toàn cục với hàm score là dot với BLEU bằng 19,02 (chênh lệch 3,98 BLEU). Khi sử dụng phương pháp Input feeding, kết quả tăng 1,21 BLEU thành 20.23 BLEU. Kết quả còn được cải thiện hơn nữa với việc sử dụng cơ chế thay thế từ hiếm (unknown replacement - unk rpl), chúng tôi đạt được điểm BLEU là 22,71 (tăng 2.43 BLEU). Kết quả của chúng tôi có một chút chênh lệch so với bài báo của Luong et al. \cite{attentionThangLuong2015} do yếu tố ngẫu nhiên của việc huấn luyện mô hình và do chúng tôi không lọc bỏ những câu có độ dài hơn 50 từ như trong bài báo thực hiện.

Các kết quả trên cho thấy những cơ chế mà chúng tôi tìm hiểu và sử dụng trong khóa luận này thực sự hiệu quả. Các cơ chế này vừa rõ ràng về lý thuyết vừa có hiệu năng tốt trong thực tế. Đặc biệt với cơ chế thay thế từ hiếm dựa vào kết quả của cơ chế Attention đã tăng kết quả của mô hình lên rất đáng kể (tăng 2,43 BLEU). Điều này cũng cho thấy tiềm năng của cơ chế Attention trong việc giải quyết bài toán Dịch máy. Không chỉ bản thân của cơ chế này nâng cao chất lượng dịch mà còn là nền tảng để những cơ chế khác sử dụng và tiếp tục nâng cao chất lượng dịch.
