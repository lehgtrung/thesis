\newpage
\chapter*{TÓM TẮT}
\addcontentsline{toc}{chapter}{TÓM TẮT} 

 Việc nghiên cứu, xây dựng các thuật toán chất lượng cao để giúp máy tính mô hình hóa, giải thích các hoạt động của con người là một vấn đề ngày càng được quan tâm và đầu tư hơn. Về tổng quan, các mô hình tự động nhận dạng hành động người có tiềm năng ứng dụng rất cao trong thực tế, như: truy vấn video, phân tích hành vi của bệnh nhân trong chẩn đoán bệnh, giám sát an ninh (chống ăn trộm, đánh nhau...), điều khiển video games thông qua cử chỉ và nhiều hệ thống tương tác ngưới-máy khác. Có thể nói, việc đưa ra một giải pháp tổng quát giúp máy hiểu được mọi cử chỉ, hành vi của con người vẫn đang là một bài toán đầy thách thức đối với cộng đồng nghiên cứu, bất chấp những nổ lực rất lớn đã được thực hiện qua hàng thập kỷ.
 
 Sự bùng nổ của công nghệ camera 3 chiều giá thành thấp (như Kinect) trong những năm gần đây đã mở ra nhiều giải pháp giúp đơn giản hóa các tác vụ nhận dạng hành động phức tạp, trong khi vẫn có thể đảm bảo được tiêu chí về tốc độ xử lý thời gian thực. Hòa nhịp cùng với xu hướng nghiên cứu hiện nay, khóa luận tập trung vào giải lớp bài toán nhận dạng hành động người trên dữ liệu đa phương thức RGB-D(gồm thông tin màu và độ sâu) thu được từ Kinect. Trên cơ sở lấy cảm hứng từ các giả thuyết về sự kích thích thị giác ở người tại các vùng nổi bật (\textit{visual attention}), khóa luận tiếp cận theo hướng khai thác các đặc trưng ngữ nghĩa, đặc thù với từng kênh dữ liệu màu-độ sâu, qua đó đề xuất một mô hình kết hợp hiệu quả các thông tin này để biểu diễn và phân lớp các hành động trên video. 

Kết quả sơ bộ mà khóa luận đạt được cho thấy việc trích chọn, kết hợp thông tin đặc trưng từ nhiều kênh dữ liệu như màu-độ sâu là rất cần thiết để tăng cường tri thức cho các hệ thống nhận dạng hành động người. Qua đó, giải pháp trình bày trong khóa luận cũng mở ra một hướng đi hứa hẹn trên con đường giúp máy tính có thể tiến gần hơn tới năng lực nhận thức và cảm thụ thị giác của con người.  
%	Hệ thị giác của người có thể cảm nhận được các cảnh liên tục, nhận biết sự vật và nắm bắt ngữ nghĩa chuyển động dễ dàng. Các nhà thần kinh, tâm lý học đã cố gắng phân tích và giải thích cơ chế giúp hệ thị giác người có thể hoạt động chính xác. Một số lý thuyết / giả thuyết như sự kích thích thị giác tại các vùng nổi bật (\textit{visual attention}), các luật "Gestalt" về cách tổ chức nhận thức đã được đặt ra, giải thích và làm sáng tỏ. Trên cơ sở lấy cảm hứng từ các thuật toán học dựa vào việc phân tích các đặc trưng thị giác nổi bật, chúng tôi cố gắng mô hình hóa và tích hợp các khám phá nhận thức thị giác quan trọng vào một hệ thống nhận dạng cử chỉ tổng quát. Đây cũng là một thành phần cơ bản, quan trọng trong mô hình tổng thể mà chúng ta đang hướng tới – một mô hình chung có thể học và hiểu mọi hoạt động, hành vi của con người.
