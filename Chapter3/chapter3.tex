\chapter{Cơ chế Attention cho mô hình Dịch máy}
\ifpdf
    \graphicspath{{Chapter3/Chapter3Figs/PNG/}{Chapter3/Chapter3Figs/PDF/}{Chapter3/Chapter3Figs/}}
\else
    \graphicspath{{Chapter3/Chapter3Figs/EPS/}{Chapter3/Chapter3Figs/}}
\fi
\label{chap_3}
\begin{quote}
\textit{Chương này trình bày về cơ chế Attention. Ở đây, chúng tôi tập trung tìm hiểu về các phiên bản của cơ chế Attention và đánh giá chúng dựa trên cơ sở Toán học. Cụ thể, chúng tôi tìm hiểu về hai phiên bản Toàn cục (Global) và Cục bộ (Local):
\begin{itemize}
	\item Toàn cục: chúng tôi nhận thấy sự hạn chế hiện có của kiến trúc Bộ mã hóa-Bộ mã hóa khi thực hiện dịch những câu dài. Do vậy, chúng tôi sử dụng cơ chế Attention phiên bản Toàn cục để giải quyết vấn đề này.
	\item Cục bộ: chúng tôi quan sát thấy rằng Attention Toàn cục vẫn còn một chút vấn đề về ý tưởng và chi phí tính toán. Với sự quan sát đó, chúng tôi hiệu chỉnh Attention Toàn cục thành phiên bản Attention Cục bộ để giải quyết những hạn chế đó.
\end{itemize}}
\end{quote}
\section{Cơ chế Attention}
Ở phần trước, chúng tôi đã trình bày về kiến trúc Bộ mã hóa-Bộ giải mã cùng với những điểm mạnh của nó trong việc giải quyết bài toán Dịch máy. Tuy nhiên, kiến trúc này vẫn còn tồn tại hạn chế về việc dịch những câu dài do những thông tin được mã hóa của câu nguồn bị mất dần theo các thời điểm về sau. Lí do mà vấn đề này tồn tại thực chất là bởi vì các mô hình LSTM được sử dụng trong Bộ mã hóa và Bộ giải mã. Bản thân mô hình LSTM chưa thật sự giải quyết hoàn toàn vấn đề "sự phụ thuộc dài hạn". Để có thể vẫn tận dụng được các mô hình LSTM mà vẫn nâng cao được chất lượng dịch, chúng tôi sử dụng cơ chế Attention.

Trước khi đi vào cách hoạt động của cơ chế Attention, chúng tôi điểm qua một chút về nguồn cảm hứng và lịch sử của cơ chế này. Cơ chế Attention được lấy cảm hứng trên cơ chế đặt sự chú ý khi quan sát sự vật, hiện tượng của thị giác con người. Khi con người quan sát một sự vật, hiện tượng nào đó bằng mắt, con người chỉ có thể tập trung vào một vùng nhất định trên sự vật, hiện tượng được quan sát để ghi nhận thông tin. Sau đó, khi cần ghi nhận thêm thông tin khác, con người sẽ di chuyển vùng tập trung lên vật thể của mắt sang vị trí khác. Những vùng lân cận xung quanh vùng tập trung sẽ bị "mờ" hơn so với vùng tập trung. Cơ chế Attention đã được ứng dụng trong lĩnh vực Thị giác máy tính từ khá lâu \cite{attentionhistory2010} \cite{attentionhistory2011}. Vào những năm gần đây, cơ chế Attention được sử dụng cho các kiến trúc mạng nơ-ron hồi quy trên bài toán Dịch máy và đã đạt được những kết quả ấn tượng.

\begin{figure}
	\centering
	\includegraphics[width=0.8\textwidth]{Attention-2}
	\caption[Minh họa cơ chế Attention.]{Minh họa cơ chế Attention. Một tầng Attention được đặt ở trước bước dự đoán đầu ra của bộ giải mã.}
	\label{fig_Attention}
\end{figure}
Cơ chế Attention được sử dụng trong đề tài này là một cơ chế sử dụng thông tin trong các trạng thái ẩn của RNN trong bộ mã hóa khi thực hiện quá trình giải mã. Cụ thể là:
\begin{itemize}
	\item Trong quá trình giải mã, trước khi dự đoán đầu ra, bộ giải mã nhìn vào các thông tin nằm trong các trạng thái ẩn của RNN ở bộ mã hóa.
	\item Ở mỗi phần tử đầu ra tại thời điểm $t$, bộ giải mã dựa vào trạng thái ẩn tại thời điểm $t$ hiện tại và quyết định sử dụng các thông tin trong trạng thái ẩn ở bộ mã hóa như thế nào.
\end{itemize}
2 phiên bản Toàn cục và Cục bộ mà trong khóa luận này chúng tôi trình bày là 2 cách mà cơ chế Attention sử dụng các trạng thái ẩn của RNN trong bộ mã hóa.
Để làm rõ hơn về ý tưởng của cơ chế Attention, dưới đây chúng tôi sẽ trình bày chi tiết về nền tảng Toán học của nó.
Attention sử dụng thêm một số đại lượng:
\begin{itemize}
	\item $a_t$: trọng số gióng hàng, $a_t$ được tính theo công thức dưới đây:
	\begin{equation}
	a_t = \text{align}(h_t, \bar{h}_s) = \frac{\exp\left(\text{score}(h_t, \bar{h}_s)\right)}{\sum_{s^{'}}\exp\left(\text{score}(h_t, \bar{h}_{s^{'}})\right)}
	\end{equation}
	$a_t$ là một véc-tơ chứa các điểm số giữa trạng thái ẩn ở thời điểm $t$ $h_t$ và các trạng thái ẩn ở câu nguồn $\bar{h}_s$. Hàm điểm số score mà chúng tôi sử dụng là gồm 2 hàm:
	\begin{equation}
	\text{score}(h_t, \bar{h}_s) = \left\{
			\begin{array}{ll}
			h^T_t\bar{h}_s \ \quad\quad dot\\
			h^T_tW_a\bar{h}_s	\quad general
			\end{array}
		\right.
	\end{equation}
	Đối với hàm score là hàm \textit{dot}, mô hình chỉ đơn giản là thực hiện tính độ tương đồng giữa 2 trạng thái ẩn. Giá trị của hàm score đạt cao nhất khi 2 véc-tơ trạng thái ẩn hoàn toàn giống nhau. Ưu điểm của hàng \textit{dot} này là chi phí tính toán thấp nên thời gian huấn luyện và suy diễn nhanh.
	Đối với hàm score là hàm \textit{general}, hàm này có sự tinh tế hơn hàm \textit{dot}. Hàm \textit{dot} thực hiện tính sự tương đồng lên tất cả cặp phần tử trong 2 véc-tơ, trong khi đó hàm \textit{general} sử dụng thêm một bộ trọng số $W_a$, do đó những thông tin giữa hai trạng thái ẩn sẽ được tính một cách chọn lọc hơn. Tuy nhiên, đổi lại thì hàm này sẽ có thời gian thực thi chậm hơn hàm \textit{dot} một chút. Trong thực tế, không có minh chứng rõ ràng nào cho thấy rằng hàm nào sẽ tốt hơn, do vậy cần phải thực nghiệm cẩn thận để có được sự lựa chọn chính xác nhất.
	\item $c_t$: véc-tơ ngữ cảnh tại thời điểm $t$, là trung bình có trọng số của các trạng thái ẩn ở câu nguồn:
	\begin{equation}
	c_t = \sum_{s}a_{ts}h_s
	\end{equation}
	Véc-tơ $c_t$ cho mô hình biết thông tin rằng với trạng thái ẩn hiện tại (chứa thông tin của quá trình dịch trước đó) thì ngữ cảnh hiện của thời điểm $t$ hiện tại là gì. Ngữ cảnh đó được thể hiện thông qua những thông tin của các trạng thái ẩn $h_s$ của câu nguồn mà được lựa chọn một cách có chọn lọc (có trọng số). Véc-tơ ngữ cảnh $c_t$ là một cách biểu diễn ngữ cảnh của ngôn ngữ đích bằng ngữ cảnh của ngôn ngữ nguồn. Trong quá trình dịch, bộ giải mã cần phải dự đoán từ tiếp theo của câu dịch. Để dự đoán được chính xác, mô hình cần phải biết được ngữ cảnh hiện tại của câu là gì. Để đảm bảo ngữ cảnh mà mô hình nhận được chính xác, mô hình không thể chỉ dựa vào các trạng thái ẩn của bộ giải mã ở các thời điểm trước đó. Do vậy, mô hình sử dụng thêm các trạng thái ẩn của các từ ở câu nguồn để thể hiện ngữ cảnh một cách chính xác hơn.
	\item $\tilde{h}_t$, véc-tơ attention tại thời điểm $t$, được tính như sau:
	\begin{equation}
	\boldsymbol{\tilde{h}_t} = \tanh(\bm{W_c}[\bm{c_t};\bm{h_t}])
	\end{equation}
	Véc-tơ attention chứa thông tin gióng hàng và trạng thái ẩn của thời điểm $t$ hiện tại. Nhờ đó, mô hình nắm giữ được nhiều thông tin hơn để có thể dự đoán tốt hơn.
\end{itemize}
Bước dự đoán đầu ra không thay đổi ngoài trạng thái ẩn $\bm{h_t}$ được thay thế bởi véc-tơ attention $\bm{\tilde{h}_t}$. $\bm{\tilde{h}_t}$ được đưa qua tầng softmax để cho ra phân bố xác suất dự đoán trên các từ:
\begin{equation}
p(y_t | y_{<t}, x) = \text{softmax}(\bm{W_s\tilde{h}})
\end{equation}
Nói một cách đơn giản, mục tiêu của cơ chế Attention là xoay quanh việc tìm véc-tơ ngữ cảnh $c_t$ một cách hiệu quả.
Tiếp theo, chúng tôi trình bày chi tiết hơn về 2 phiên bản Toàn cục và Cục bộ. 2 phiên bản này chỉ khác nhau về cách suy ra véc-tơ ngữ cảnh $\bm{c_t}$, còn các bước còn lại giống nhau.
Quy trình tính toán của cơ chế Attention: $h_t -> a_t -> c_t -> \tilde{h}_t$
\section{Attention Toàn cục}
Ý tưởng của Attention toàn cục là nhìn vào toàn bộ các vị trí nguồn (các trạng thái ẩn của RNN ở bộ mã hóa) khi thực hiện giải mã.
Khi đó trọng số gióng hàng $a_t$ là một véc-tơ có kích thước thay đổi và bằng số trạng thái ẩn (số từ) ở câu nguồn: $\text{len}(a_t) = S$.

\begin{equation}
a_t = \text{align}(h_t, \bar{h}_s) = \frac{\exp\left(\text{score}(h_t, \bar{h}_s)\right)}{\sum^{S}_{s^{'}=1}\exp\left(\text{score}(h_t, \bar{h}_{s^{'}})\right)}
\end{equation}

\begin{figure}
	\centering
	\includegraphics[width=0.8\textwidth]{Global-Attention_2.png}
	\caption[Minh họa cơ chế Attention Toàn cục.]{Minh họa cơ chế Attention Toàn cục. Tại thời điểm $t$, bộ giải mã nhìn vào toàn bộ trạng thái ẩn ở các vị trí nguồn.}
	\label{fig_Global_Attention}
\end{figure}

Ưu điểm của phương pháp này là ý tưởng đơn giản, dễ cài đặt nhưng vẫn đạt được hiệu quả tốt (sẽ được trình bày ở phần thực nghiệm). Tuy nhiên, ý tưởng này vẫn còn chưa thực sự tự nhiên và còn hạn chế. Khi dịch một từ thì không cần phải đặt "sự chú ý" lên toàn bộ câu nguồn, chỉ cần đặt "sự chú ý" lên một số từ cần thiết. Mặc dù khi mô hình Attention Toàn cục được huấn luyện tốt thì hoàn toàn có thể chỉ đặt "sự chú ý" lên một số từ thật sự cần thiết, nhưng dễ thấy rằng bản thân mô hình vẫn phải tiêu tốn chi phí cho việc tính toán trọng số gióng hàng $a_t$ cho những vị trí không cần thiết. Đó là trường hợp lý tưởng cho mô hình Attention Toàn cục, nhưng trong thực tế, để đạt được độ chính xác như thế thì phải tiêu tốn nhiều tài nguyên cho việc huấn luyện mô hình như tài nguyên về tập dữ liệu đủ lớn, đủ tốt hay thời gian huấn luyện phải đủ lâu.
Để giải quyết hạn chế trên của Attention Toàn cục, chúng tôi đã tìm hiểu và sử dụng phiên bản tinh tế hơn, đó là mô hình Attention Cục bộ. Ở phần tiếp theo, chúng tôi sẽ trình bày về mô hình này.

\section{Attention Cục bộ}
Như đã nêu ở phần trước, Attention Toàn cục có một hạn chế là đặt "sự chú ý" lên toàn bộ các từ ở câu nguồn khi dịch từng từ ở câu đích. Điều này gây tiêu tốn chi phí tính toán và có thể tạo ra những câu dịch không thực tế khi dịch những câu dài như trong các đoạn văn hay trong một tài liệu. Attention Cục bộ ra đời để giải quyết hạn chế này.

Khi dịch mỗi từ ở câu đích, Attention Cục bộ chỉ đặt "sự chú ý" lên một số từ gần nhau ở câu nguồn. Mô hình này lấy cảm hứng từ sự đánh đổi giữa 2 mô hình "soft attention" và "hard attention" được đề xuất trong công trình Show, Attend and Tell \cite{showattendandtellXu2015} để giải quyết bài toán Phát sinh câu miêu tả cho ảnh (Image Captioning). Trong công trình \cite{showattendandtellXu2015}, Attention Toàn cục tương ứng với "soft attention", "sự chú ý" được đặt trên toàn bộ bức ảnh. Còn "hard attention" thì đặt "sự chú ý" lên một số phần của bức ảnh.

Dễ thấy, với cách hoạt động chỉ tập trung một số các từ gần nhau ở câu nguồn, mô hình hoạt động gần với cách con người tập trung vào một sự vật, hiện tượng nào đó. Chi phí cho huấn luyện và dự đoán sẽ được giảm bớt bởi vì chúng ta chỉ thực hiện tính véc-tơ trọng số gióng hàng $a_t$ cho những từ mà mô hình đặt "sự chú ý" lên.

Để làm rõ hơn về cách thức hoạt động của mô hình Attention Cục bộ, chúng tôi sẽ trình bày cụ thể hơn về nền tảng Toán học của mô hình này. Bên cạnh những đại lượng đã có ở mô hình Attention Toàn cục, Attention Cục bộ có thêm và thay đổi một số đại lượng như sau:
\begin{itemize}
	\item $p_t$: vị trí đã được gióng hàng. Tại mỗi thời điểm $t$, mô hình sẽ phát sinh một số thực $p_t$. Số thực này có giá trị nằm trong đoạn $[0, S]$ với ý nghĩa rằng đây là vị trí đã được gióng hàng của với từ ở câu nguồn tại thời điểm $t$ hiện tại. Hay nói cách khác, "sự chú ý" được đặt trên từ có vị trí $p_t$ này. Để ý thấy rằng có sự không tự nhiên khi $p_t$ là một số thực, do vậy $p_t$ không thể cho biết được chính xác từ nào sẽ được đặt "sự chú ý" lên. Thực tế, với miền giá trị số thực, $p_t$ có tác dụng là dùng để làm vị trí trung tâm cho các từ lân cận. Để làm rõ hơn về vấn đề này, chúng tôi sẽ trình bày rõ ràng hơn ở sau.
	\item Đối quá trình tính véc-tơ ngữ cảnh $c_t$ có sự thay đổi rằng mô hình xét các vị trí ở câu nguồn mà nằm xung quanh vị trí $p_t$ một đoạn $D$. $D$ là một đại lượng với miền số nguyên lớn hơn 0 và được gọi là kích thước cửa sổ. Cụ thể:
	\begin{equation}
	c_t = \sum_{x \in [p_t - D, p_t + D]} a_{tx}\tilde{h}_x
	\end{equation}
	$D$ là một siêu tham số của mô hình. Việc lựa chọn giá trị của $D$ là dựa vào thực nghiệm. Theo đề xuất của \cite{attentionThangLuong2015}, chúng tôi lựa chọn $D = 10$.
\end{itemize}

\begin{figure}
	\centering
	\includegraphics[width=0.8\textwidth]{Local-Attention_2.png}
	\caption[Minh họa cơ chế Attention Cục bộ.]{Minh họa cơ chế Attention Cục bộ. Tại thời điểm $t$, bộ giải mã nhìn vào một số trạng thái ẩn ở các vị trí nguồn.}
	\label{fig_Local_Attention}
\end{figure}

Mô hình Attention Cục bộ có 2 biến thể:
\begin{itemize}
	\item Gióng hàng đều (monotonic alignment - local-m): vị trí được gióng hàng được phát sinh một cách đơn giản bằng cách cho $p_t = t$ tại mỗi thời điểm $t$. Ta giả định rằng các từ ở câu nguồn và các từ ở câu đích được gióng hàng đều nhau theo từng từ.
	\item Gióng hàng dự đoán (predictive alignment - local-p): giả định rằng tất cả từ ở câu nguồn và câu đích đều được gióng hàng đều nhau không thực tế vì giữa 2 ngôn ngữ có ngữ pháp riêng và trật tự từ khác nhau. Chúng tôi sẽ trình bày rõ ràng hơn vào các phần sau. Do vậy, mô hình sẽ phát sinh vị trí được gióng hàng $p_t$ một cách tự nhiên hơn cho phù hợp đặc điểm của ngôn ngữ. Cụ thể mô hình sẽ phát sinh vị trí $p_t$ tại mỗi thời điểm $t$ như sau:
	\begin{equation}
	p_t = S \cdot \text{sigmoid} (v^T_p \tanh(W_p h_t))
	\end{equation}
	Trong đó, $v_p$ và $W_p$ là 2 tham số mới của mô hình cho việc dự đoán vị trí $p_t$. Mô hình cần học 2 tham số này. Miền giá trị của $p_t \in [0, S]$.
	Để "ưu tiên" các vị trí được gióng hàng $p_t$, mô hình thêm vào trọng số gióng hàng của những từ lân cận đó một lượng có giá trị bằng giá trị của phân phối chuẩn (Gauss) mà đã được đơn giản hóa với trung bình $p_t$ và độ lệch chuẩn $\sigma = \frac{D}{2}$:
	\begin{equation}
	p_t = \text{align}(h_t, \bar{h}_s)\exp\left(-\frac{(s-p_t)^2}{2\sigma^2}\right)
	\end{equation}
	Mô hình sử dụng hàm gióng hàng như các phiên bản trước. $s$ là giá trị số nguyên thể hiện các vị trí nằm xung quanh $p_t$ mà nằm trong cửa sổ $D$.
\end{itemize}
 Đối với những vị trí $s$ nằm ngoài câu (cửa sổ $D$ vượt qua các biên của câu) thì mô hình sẽ bỏ qua những vị trí $s$ nằm ngoài và chỉ xem xét những vị trí $s$ nằm trong biên của câu.
 
 Véc-tơ trọng số gióng hàng $a_t$ ở Attention Cục bộ có kích thước cố định $\in \mathbb{R}^{2D + 1}$ và thường ngắn hơn $a_t$ ở Attention Toàn cục. Local-p và local-m giống nhau chỉ khác rằng local-p tính vị trí $p_t$ một cách linh hoạt và sử dụng một phân phối chuẩn đã được đơn giản hóa để điều chỉnh các trọng số gióng hàng gốc $\text{align}(h_t, \bar{h}_s)$. Việc sử dụng thêm phân phối chuẩn để khuyến khích mô hình đặt "sự chú ý" lên vị trí $p_t$ và phân chia dần cho các vị trí lân cận. Nếu không có việc sử dụng phân phối chuẩn này, mô hình có thể sẽ đặt "sự chú ý" hoàn toàn lên các từ lân cận xung quanh $p_t$ mà không phải là vị trí $p_t$. Điều này không phù hợp với ý tưởng ban đầu của việc phát sinh vị trí $p_t$.
Với cơ chế được trình bày cụ thể như trên, mô hình Attention Cục bộ hoạt động tự nhiên hơn, phù hợp với cách con người đặt "sự chú ý" khi quan sát sự vật, hiện tượng. Bên cạnh đó, Attention Cục bộ giảm chi phí tính toán của mô hình.

\section{Phương pháp Input feeding}
Trong quá trình dịch, các mô hình được đề cập ở trên như Attention Toàn cục hay Cục bộ, đều vẫn còn một hạn chế về cách đặt "sự chú ý" hay gióng hàng lên các vị trí nguồn. Ở mỗi thời điểm $t$ khi dịch một từ ở câu đích, việc đặt "sự chú ý" của thời điểm $t$ độc lập hoàn toàn với việc đặt "sự chú ý" ở các thời điểm trước đó. Việc quyết định gióng hàng như thế nào (véc-tơ $a_t$) hoàn toàn phụ thuộc vào điểm số (giá trị của hàm score) giữa trạng thái ẩn $h_t$ hiện tại và các trạng thái ẩn $\bar{h}_s$ ở câu nguồn. Trong thực tế, khi dịch, một từ ở câu nguồn chỉ tương ứng với một vài từ ở câu đích. Do vậy, mô hình cần phải theo dõi xem là những từ nào ở câu nguồn đã được dịch trước đó thì hạn chế đặt "sự chú ý" lên lại những từ đó. Việc không có cơ chế kiểm soát những từ nào đã được dịch sẽ khiến cho mô hình sẽ rơi vào 2 trường hợp "được dịch quá nhiều" (over-translated) hoặc "được dịch quá ít" (under-translated). Tức là có một số từ ở câu nguồn sẽ được đặt "sự chú ý" lên quá nhiều lần dẫn tới bỏ qua những từ quan trọng khác hoặc là một số từ quan trọng được đặt "sự chú ý" lên quá ít dẫn tới việc bỏ qua thông tin của từ đó trong quá trình dịch. Dù là trường hợp nào thì cũng gây giảm chất lượng dịch của mô hình.

Trong Dịch máy Thống kê, Koehn et al. 2003 \cite{smtKoehn2003} đã đề xuất một mô hình dịch dựa trên cụm từ (phrase-based) mà có cơ chế để giải quyết vấn đề trên. Cơ chế này rất đơn giản và trực quan. Trong quá trình dịch, bộ giải mã duy trì một véc-tơ bao phủ (coverage vector) để chỉ ra rằng từ ở câu nguồn nào đã được dịch hoặc chưa được dịch. Quá trình dịch được hoàn thành khi toàn bộ từ ở câu nguồn được "bao phủ" hay dẵ được dịch. Trong khi đó, các mô hình Dịch máy Nơ-ron hiện nay chỉ kết thúc quá trình dịch khi và chỉ khi gặp kí tự kết thúc câu hoặc vượt quá số lượng từ cho trước. Việc này dễ dẫn đến trường hợp "được dịch quá nhiều" khi kí hiệu kết thúc câu xuất hiện trễ hay ngược lại dẫn đến trường hợp "được dịch quá ít" khi kí hiệu kết thúc câu xuất hiện sớm. Ngoài ra còn bị ảnh hưởng bởi số lượng từ quy định khi dịch.

Công trình \cite{attentionThangLuong2015} đề xuất một cơ chế góp phần giải quyết vấn đề ở trên:  (tạm dịch là "cho đầu vào ăn" // TODO: dịch khác). Ý tưởng và cách thực hiện của Input feeding rất đơn giản. Nhận thấy véc-tơ attention $\tilde{h}_{t-1}$ lưu giữ thông tin gióng hàng của thời điểm $t-1$ trước đó, mô hình thực hiện truyền véc-tơ $\tilde{h}_{t-1}$ vào đầu vào $x_t$ của thời điểm $t$ hiện tại. Bằng cách như vậy, mô hình có thể nắm được thông tin gióng hàng trước đó từ $\tilde{h}_{t-1}$. Cụ thể, véc-tơ $\tilde{h}_{t-1}$ được nối với véc-tơ đầu vào của thời điểm $t$ là $x_t$:
\begin{equation}
x^{'}_t = [x_t, \tilde{h}_t]
\end{equation}

\begin{figure}
	\centering
	\includegraphics[width=0.8\textwidth]{Input-feeding_2.png}
	\caption[Minh họa cơ chế Attention Cục bộ.]{Minh họa phương pháp Input feeding. Tại thời điểm $t$, bộ giải mã nhận đầu vào gồm véc-tơ attention ở thời điểm trước đó $t-1$ và từ hiện tại $x_t$.}
	\label{fig_Input_feeding}
\end{figure}
Tuy nhiên, phương pháp này chưa thực sự giải quyết triệt để vấn đề "được dịch quá nhiều" hay "được dịch quá ít". Vì mô hình chỉ nhận được thông tin gióng hàng từ các thời điểm trước đó nhưng lại không được hướng dẫn, ràng buộc cụ thể nào mà có thể giải quyết vấn đề này. Việc giải quyết vấn đề trên hoàn toàn phụ thuộc vào quyết định của mô hình . Mặc dù chưa thực sự giải quyết triệt để, nhưng lại cho mô hình tăng thêm tính mềm dẻo trong việc sử dụng thông tin gióng hảng trước đó. Trong thực tế, phương pháp này đã cải thiện chất lượng dịch lên đáng kể.

Ngoài ra, phương pháp này giúp cho mô hình phức tạp hơn nhờ vào việc đưa véc-tơ attention $\tilde{h}_{t}$ vào đầu vào của thời điểm tiếp theo, đồng thời làm tăng khả năng học của mô hình.

\section{Kĩ thuật thay thế từ hiếm}
Trong quá trình dịch thuật, có rất nhiều hạn chế gây ảnh hưởng tới chất lượng của bản dịch. Trong phần này, chúng tôi đề cập tới một vấn đề quan trọng mà dù là con người hay máy tính đều gặp phải và rất khó giải quyết. Đó là vấn đề về những "từ hiếm" (unknown words). 

Mỗi ngôn ngữ có muôn hình vạn trạng các từ ngữ khác nhau. Số lượng từ ngữ trong một ngôn ngữ là không có định. Trong quá trình hình thành và phát triển ngôn ngữ, theo thời gian số lượng từ ngữ sẽ tăng lên hoặc mất đi (bị lãng quên hay không dùng nữa) tùy thuộc vào hoàn cảnh, môi trường sử dụng của ngôn ngữ đó. Nhưng thường đối với những ngôn ngữ phổ biển hiện nay thì số lượng từ ngữ tăng lên lớn hơn nhiều so với số lượng từ ngữ mất đi. Khi xã hội phát triển, nhu cầu giao tiếp giữa các dân tộc, quốc gia, nền văn hóa khác nhau cũng tăng theo. Mỗi nơi lại có cách sử dụng ngôn ngữ khác nhau, do đó bộ từ vựng của mỗi ngôn ngữ cũng phải thay đổi sao cho phù hợp với nhu cầu giao tiếp. Khoa học kĩ thuật phát triển kèm theo đó là những khám phá về thế giới tự nhiên. Những sự vật, hiện tượng mới được phát hiện ngày càng nhiều. Và không phải sự vật, hiện tượng nào cũng có thể được mô tả, thể hiện bằng những vốn từ vựng vốn có của một số ngôn ngữ. Ngoài ra còn có nhiều lí do làm cho bộ từ vựng của các ngôn ngữ thay đổi theo thời gian.

Với tốc độ phát triển của ngôn ngữ là như vậy nhưng khả năng của con người là hữu hạn. Một người dù có thông thạo một ngôn ngữ tới đâu thì cũng không thể nào biết được hết tất cả từ vựng của ngôn ngữ đó. Theo thống kê, số lượng từ ngữ cần để giao tiếp hàng ngày trong tiếng Anh chỉ khoảng từ 2000-3000 từ, đối với lĩnh vực chuyên ngành thì khoảng 5000-6000 từ. Nhưng theo kích thước của một số bộ từ điển thịnh hành trong tiếng Anh thì số lượng từ vựng của những bộ từ điển đó khoảng 60000 từ. Tức là đa số mọi người chưa biết hết được 10\% từ vựng của tiếng Anh. Do vậy khi thực hiện việc dịch thuật giữa các ngôn ngữ với nhau, mọi người chỉ có thể dịch tốt khi văn bản, hội thoại cần dịch thuộc về chủ đề mà họ quen thuộc. Mọi người sẽ gặp khó khăn khi gặp những từ nằm ngoài bộ từ vựng của bản thân (out-of-vocabulary words - OOV words) vì không biết phải dịch như thế nào.

Khi huấn luyện một mô hình Dịch máy thì cần phải có một bộ từ vựng cố định cho mô hình đó trong suốt quá trình huấn luyện và dự đoán. Kích thước của bộ từ vựng này bị hạn chế với số lượng nhất định. Sự hạn chế về kích thước này xuất phát từ nhiều lí do như giới hạn về dữ liệu huấn luyện, khả năng học của mô hình, tài nguyên tính toán (phần cứng), v.v... Do vậy việc quyết định xem những từ nào sẽ được đưa vào bộ từ vựng của mô hình cũng rất quan trọng. Thông thường có 2 chiến thuật để xây dựng bộ từ vựng này. Cách đầu tiên phù hợp cho việc phát triển các ứng dụng là lấy các từ vựng có trong dữ liệu huấn luyện làm bộ từ vựng và lọc ra những từ nào có tần số xuất hiện trong dữ liệu huấn luyện thấp hơn một ngưỡng nhất định (ví dụ: lọc ra những từ vựng nào có tần số xuất hiện ít hơn 10). Cách thứ 2 thường phù hợp cho việc nghiên cứu, đó là lựa chọn số lượng từ vựng nhất định mà có tần số xuất hiện cao nhất (ví dụ: lấy 50000 từ có tần số xuất hiện cao nhất). Do đó có những từ xuất hiện trong dữ liệu huấn luyện nhưng vì có tần số xuất hiện thấp nên bị coi là từ nằm ngoài bộ từ vựng (OOV). Đó là lí do chúng tôi gọi đây là vấn đề "từ hiếm".

Có nhiều cách để giải quyết vấn đề này, cách mà mọi người hay sử dụng nhất là thêm từ mới đó vào bộ từ vựng. Cách thứ 2 là giữ nguyên từ đó và đưa nó vào vị trí thích hợp trong câu ở ngôn ngữ đích. Trong khóa luận này chúng tôi sẽ sử dụng cách thứ 2 để giải quyết vấn đề các từ nằm ngoái bộ từ vựng.

Kĩ thuật thay thế từ hiếm mà chúng tôi trình bày sau đây là một phương pháp dựa trên kết quả của cơ chế Attention. Do vậy, hiệu quả của phương pháp này phụ thuộc lớn vào độ chính xác của cơ chế Attention. Kĩ thuật này chúng tôi sử dụng từ bài báo của Jean et al., 2015 \cite{JeanUnkRepl} về sử dụng cơ chế Attention trong mô hình Dịch máy nơ-ron. Nếu mô hình không sử dụng cơ chế Attention thì cũng không sử dụng được phương pháp thay thế từ hiếm này. Kĩ thuật này chỉ được sử dụng trong quá trình dự đoán, trong quá trình huấn luyện thì không sử dụng. Cách hoạt động của phương pháp này rất đơn giản. Sau khi mô hình đã dự đoán (dịch) xong một câu, mô hình sẽ thực hiện xử lý những từ nào mà được dự đoán là từ hiếm (unknown words) trong câu đã được dự đoán (những từ hiếm được ký hiệu là \textit{<unk>}). Đối với mỗi từ hiếm, mô hình sẽ thực hiện dịch lại từ đó bằng cách chọn một từ phù hợp trong câu nguồn rồi thực hiện sao chép từ được chọn vào từ hiếm hiện tại. Cách mà mô hình lựa chọn từ phù hợp là dựa vào véc-tơ trọng số gióng hàng $a_t$. Mô hình sẽ lựa chọn từ nào có trọng số cao nhất.

\begin{figure}
	\centering
	\includegraphics[width=0.5\textwidth]{unk-rpl-example.png}
	\caption[Minh họa kĩ thuật thay thế từ hiếm.]{Minh họa phương pháp thay thế từ hiếm. Khi gặp một từ hiếm (được kí hiệu là <unk>) trong kết quả dự đoán, mô hình sẽ tìm một từ ở câu nguồn có trọng số gióng hàng từ kết quả cơ chế Attention cao nhất và thực hiện sao chép từ đó thay cho từ hiếm hiện tại. (Mũi tên càng đậm thì trọng số gióng hàng càng cao). Kết quả dự đoán được cập nhật với từ hiếm đã được thay thế. }
	\label{fig_unk_rpl_example}
\end{figure}

Với kĩ thuật đơn giản là tận dụng ý nghĩa của kết quả của cơ chế Attention, kĩ thuật này đã cải thiện kết quả dịch lên một cách rõ rệt (sẽ được trình bày ở trong phần thực nghiệm).


